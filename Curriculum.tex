\documentclass[a4paper,10pt]{article}
\usepackage[margin=1in]{geometry}
\usepackage{enumitem}
\usepackage[hidelinks]{hyperref}
\usepackage{titlesec}

\titleformat{\section}{\large\bfseries}{}{0em}{}[\titlerule]

\begin{document}

\begin{center}
    {\LARGE \textbf{PABLO GÓMEZ VILLOUTA}} \\
    Santiago, Chile - \href{mailto:pablofgomez@alumnos.uai.cl}{pablofgomez@alumnos.uai.cl} - (+56) 9 9219 3809 - \href{https://www.linkedin.com/in/pablo-g%C3%B3mez-villouta-601868288/}{LinkedIn} - \href{https://github.com/PabloGomez-97}{Github}
\end{center}

\section*{PERFIL PROFESIONAL}
Soy una persona apacionada por la Ciberseguridad, Data Science y la Docencia, motivado por el aprendizaje y la enseñanza. Disfruto compartir conocimientos y colaborar en proyectos interdisciplinarios, adaptándome a nuevas tecnologías y metodologías ágiles. Mi objetivo es seguir desarrollando mis habilidades y contribuir a iniciativas innovadoras en estas áreas.

\section*{EXPERIENCIA}
\textbf{AYUDANTÍA CURSOS INFORMÁTICOS, UAI}, Peñalolén, Santiago \\
\textit{Profesor Part-Time} \hfill Marzo 2024 – Actualidad
\begin{itemize}[noitemsep]
    \item Desarrollo clases de Redes de Computadores y Bases de Datos en la Universidad Adolfo Ibáñez como Ayudante, entregando contenidos relevantes de la actualidad, siendo Modelo OSI, IPv4, IPv6, Modelamiento, Redes Inalámbricas, Soluciones Cloud Computing y en Bases de Datos, programas de modelamiento de datos, SQL, NoSQL, API’s, JSON, Big Data, Web Services, entre otros.
    \item Entrego contenido a diferentes secciones de Redes de Computadores y Bases de Datos con metodologías teóricas y prácticas en entornos seguros (Sandbox) en CISCO Packet Tracer, Python, MongoDB, entre otros.
\end{itemize}
\noindent
\textbf{INMOTION}, Las Condes, Santiago \\
\textit{Practicante como Desarrollador} \hfill Enero 2024 – Abril 2024
\begin{itemize}[noitemsep]
    \item Desarrollé una aplicación web en Oracle APEX para la gestión de riesgos en la empresa, enfocándose netamente en el área de la Ciberseguridad.
    \item Creé una aplicación web que integra la Ciberseguridad y los diferentes métodos como la triada CID con la gestión de riesgos que se podían presentar en la empresa.
\end{itemize}

\section*{PROYECTOS PERSONALES}
\textbf{EDUCAI SOLUTIONS} (\textit{React, Express, MongoDB}) \hfill Marzo 2024
\begin{itemize}[noitemsep]
    \item Desarrollé el Front-End de un CRM con Inteligencia Artificial (chatbot) como parte de un curso de la universidad.
\end{itemize}
\textbf{ANALITIKS SpA} (\textit{Python, Twilio, Flask}) \hfill Agosto 2024
\begin{itemize}[noitemsep]
    \item Desarrollé un chatbot con Inteligencía Artificial para la empresa Analitiks Spa, con el fin de automatizar la atención al cliente.
\end{itemize}

\section*{COMPETENCIAS TÉCNICAS}
\begin{itemize}[noitemsep]
    \item \textbf{Lenguajes de Programación:} Python, R, SQL JavaScript, Oracle APEX, HTML, CSS
    \item \textbf{Tecnologías:} Node, React, Azure, Pandas, Git, Docker, Numpy, Excel
    \item \textbf{Idiomas:} Español Nativo, Inglés Básico
\end{itemize}

\section*{EDUCACIÓN}
\textbf{UNIVERSIDAD ADOLFO IBÁÑEZ}, Santiago, Chile \\
\textit{Ingeniería Civil Informática} \hfill Marzo 2021 – Noviembre 2025

\section*{CURSOS Y CERTIFICACIONES}
\begin{itemize}[noitemsep]
    \item \textbf{Microsoft:} Azure Data Fundamentals DP-900, Azure AI Fundamentals AI-900, Azure Fundamentals AZ-900
    \item \textbf{Udemy:} Curso Completo de Ciberseguridad Defensiva, Santiago Hernández
\end{itemize}

\end{document}
